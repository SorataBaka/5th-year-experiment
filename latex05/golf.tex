\documentclass[a4j, twocolumn]{jarticle}

\usepackage[dvipdfmx]{graphicx}
\usepackage{graphicx}

% chktex-file 44


\begin{document}
\title{ゴルフについて基礎知識}
\author{Christian Harjuno\thanks{釧路工業高等専門学校情報工学科}}
\date{2025年4月28日}
\maketitle

\section{ゴルフの誕生}
世界のスポーツの中でも, 最も古い歴史を持つと言っても通信ではないのが, ”ゴルフ”である. そして, 長い歴史も持っているために, その起源はいまだに謎に包まれている. 最も有力とされるのは, スコットランドの羊飼いが先に曲がった杖で小石や羊の糞打ち, ウサギの巣穴へ入れて遊んだのがゴルフにつながったという説である. 15世紀にはすでにゴルフが広く多くの人たちの間で楽しまれていた. しかし, 

\begin{itemize}
  \item 兵士がゴルフに夢中になり, 訓練をおろそかにしたため, イングランドとの戦争に負けた.
  \item 国民の多くがゴルフばかりしていて働かなくなった.
\end{itemize}
などの理由から, 1457年, スコッとランド王ジェームス2世が, 第14議会で「フットボールとゴルフは絶対禁止しなくてはならない.」という, 「ゴルフ禁止令」を発令したという記事が残っている\cite{JINSEI}.

その後, 数回に渡り, 「ゴルフ禁止令」が発令された. しかし, 1502年当時の国王シェームス4世は, 自分がゴルフをしたかったために, このゴルフ禁止令を解消した. そして, 人々は再びゴルフを楽しむようになったのである. お隣のイングランドへも, この頃にゴルフが伝わったとされている. 

\section{ゴルフの道具}
ゴルフをするにはなんといってもクラブとボールがなければ始まらない. クラブとボールの規格はルールで厳しく制限され, 選手は14本のクラブまで使用して良いことになっている\cite{GOLFRULE}. このクラブセッティングが選手の成績に大きく影響する. 表\ref{wood-data}, \ref{iron-data}に主なクラブの名称, 飛距離などを記述する.
\begin{table}[htb]
  \caption{表 1: ウッドの種類}\label{wood-data}
  \begin{center}
    \begin{tabular}{|c|c|c|}
      \hline
      \multicolumn{3}{|c|}{ウッド(Wood)} \\
      \hline
      \hline
      番手 & 名称 & 飛距離 \\
      \hline
      1番 & ドライバ & 300 \\
      \hline
      2番 & ブラッシ & 280 \\
      \hline
      3番 & スプーン & 270 \\
      \hline
      4番 & バッフィ & 260\\
      \hline 
      5番 & クリーク & 250 \\
      \hline
      
    \end{tabular}
  \end{center}
\end{table}
\vspace{-3em}
\begin{table}[htb]
  \caption{表 2: アイロンの種類}\label{iron-data}
  \begin{center}
    \begin{tabular}{|c|c|c|}
      \hline
      \multicolumn{3}{|c|}{アイロン (Iron)} \\
      \hline
      \hline
      番手 & 名称 & 飛距離 \\
      \hline
      1番 & ドライビングアイロン & 250 \\
      \hline
      2番 & ミッドアイロン & 240 \\
      \hline
      3番 & ミッドマッシー & 230 \\
      \hline
      4番 & マッシーアイロン & 220\\
      \hline 
      5番 & マッシー & 210 \\
      \hline
      6番 & スペードマッシー & 200 \\
      \hline
      7番 & マッシーニブリック & 190\\
      \hline 
      8番 & ピッチングニブリック & 180 \\
      \hline
      9番 & ニブリック & 170 \\
      \hline
    \end{tabular}
  \end{center}
\end{table}
\vspace{-2em}

\begin{description}
  \item[ドライバー] その昔、材質に柿木(パーシモン)をしそうしていたことからウッド(Wood)とも呼ばれている. 芯に当たった時は大きな距離が出るが、その反面左右に曲がりやすい. 
  \item[フェアウイウッド] ドライバー以外のウッド. 最近はアイアンの代わりに利用するゴルファーも多い. 片山晋悟がフェアウエイウッドの名手とされている. しかし、ちょっとジジくさい.   
  \item[アイアン] 文字通り鉄(iron)でできている. アイアンといってもステンレス、チタンなど固くて軽い材料が利用されている. 1番から9番まで番手があるが、アマチュアは1、2、3番アイアンは難しくで使えない. アイアンは距離を合わせる場合やグリーンを狙う時に使われる. 
  \item[ウエッジ] 主にグリーン周りやバンカーに入った球を打つときに使う. 距離が出ないが、弾道が高く球が止まりやすい性質がある. ウエッジブランドではボーケイ等が有名.
  \item[パター] グリーンに乗った球をカップに沈めるときに使う. とにかくセンスが必要. この一打で優勝の行方が変わる.
\end{description}

\section{ドライバーの飛距離}
ドライバーの飛距離はトッププロだと300ヤードを楽に超える。飛距離はヘッドスピードとクラブの反発係数に依存する。距離を$D$, ヘッドスピードを$S$, 打ち出し角度を\theta, クラブの反発係数を$R$とすると 式のような関係式が成り立つ\cite{GOLFRULE}
\begin{thebibliography}{99}
  \bibitem{JINSEI} 穴掘好雄, ''俺のゴルフ人生'', OB出版, 2016.
  \bibitem{GOLFRULE} M. Jackson and J. Madonna, ``Official Golf Rule'', Sports science, 2011 
\end{thebibliography}

\end{document}

% chktex 44