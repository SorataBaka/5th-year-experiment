\documentclass[a4j]{jarticle}

\begin{document}

\title{数式の表記の仕方}
\author{おじゃる丸\thanks{Eテレ株式会社} \and 野原しんのすけ\thanks{春日部市役所}}
\date{}

\maketitle

\LaTeX で文章中に数式を記述すると$\sqrt{a^2-b}$のようになる.

別行立てで数式をかくと\[\int_0^\infty\frac{\sin x}{\sqrt{x}}dx = \sqrt{\frac{\pi}{2}}\]のようになる.

さらに、各数式に番号をつけると
\begin{equation} \label{eq1}
  \sqrt{a^2-b}
\end{equation}

\begin{equation} \label{eq2}
  \int_0^\infty\frac{\sin x}{\sqrt{x}}dx = \sqrt{\frac{\pi}{2}}
\end{equation}
となる.

上の (\ref{eq1})~式は簡単だが, (\ref{eq2})~式は複雑だ.

\vspace{2zh}

集合は$A=\{\ 2i \times 5 \mid 1 \le i \leq n\}$のように書くとおしゃれだ.
対数も$log_{10} n$は超ダッサ. 正しくは$\log_{10} n$と書く.
sin, cos, max, min も見た目は同じだが数式コマンドを使って, $\sin$, $\cos$, $\max$, $\min$と書くべきだ.
例えば, xをイタリック体にするとき, \textit{x} とする代わりに$x$と書いた方が楽だ. でも, $I have a pen$はおかしくなるので, \textit{I have a pen}を使う.

最後に, 数式でも文中でもそうだが, 半角の`,'を使うときは, 必ず直後に半角スペースを入れることを習慣にせよ.
He, 185cm, She, 168cm は格好いいが, He,185cm,She,168cm はダサい.

\end{document}