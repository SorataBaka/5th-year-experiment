\documentclass[a4j, twocolumn]{jarticle}
\usepackage{graphicx}
\usepackage{titling}
\graphicspath{{./images/}}

\begin{document}
  \title{インドネシア \\\large 東南アジアの隠れた美しさ}
  \author{Christian Harjuno\thanks{釧路工業高等専門学校情報工学科}}
  \date{\today}
  \maketitle
  \section{インドネシアってなんですか?}
  インドネシア又はインドネシア共和国は東南アジアにある一つの国です. シンガポール, マレーシア, オーストラリアに間にあり, 面積は約$190.000km^2$で, インドネシアは世界で14番目に大きい国土面積を持っています. インドネシアは多数の島々を有することで知られており, 周囲を海に囲まれ, 17,508以上の島が存在する. \cite{ANDREFOUET2022104848}
  2020年の国勢調査によると, インドネシアの人口は約2億7,000万人で, 世界で4番目に多いです.\cite{unstats2023} その地理的条件と歴史的背景から, インドネシアには1,331以上の公的に認定された民族集団が存在し, 700以上の言語が国全体にわたって分布している。


  % \begin{thebibliography}{99}
  %   \bibitem{UNYEARBOOK} 国連統計局. (2023). 国際人口および社会統計年鑑 第74巻, 58ページ. 国連.
  % \end{thebibliography}
  \bibliographystyle{plain}
  \bibliography{citations}
\end{document}